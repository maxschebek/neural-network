\documentclass{article}
\usepackage[utf8]{inputenc}
\usepackage{amsmath}
\setlength\parindent{0pt}
\setlength{\parskip}{5pt}%

\title{Neural network with backpropagation}

\begin{document}

\maketitle




\subsection*{Forward propagation}
We consider a network consisting of $L$ layers: one input layer, one output layer, and $L$-$2$ hidden layers.  A given layer $j$ of the network with $s_j$ units can be represented in terms of two vectors, $\mathbf{z}^{(j)}$ and $\mathbf{a}^{(j)}$, whereas the propagation of data from the input to the output layers is given in terms of $L$-$1$ weight matrices $\boldsymbol{\Theta}$.

For a given input instance with feature vector $\mathbf{x}$, for the first layer ($j=1$) the vector $\mathbf{a}$ is set as

\begin{equation}
\mathbf{a}^{(1)} = (1,\mathbf{x})^{\rm T}.
\end{equation}

For $L > j > 1$, $\mathbf{z}^{(j)}$ and $\mathbf{a}^{(j)}$ can be obtained following

\begin{equation}
\mathbf{z}^{(j)} = \boldsymbol{\Theta}^{(j-1)}a^{(j-1)}
\end{equation}
with

\begin{equation}
\mathbf{a}^{(j)} = \left(1,g(\mathbf{z}^{(j)})\right)^{\rm T}=\left(1, g(z^{(j)}_1),g(z^{(j)}_2),...,g(z^{(j)}_{s_j})\right)^{\rm T}.
\end{equation}
The activation function $g$ is given by the sigmoid function

\begin{equation}
g(z) = \frac{e^z}{1+ e^z}.
\end{equation}

The last layer does not have a bias unit, such that the output layer is given by

\begin{equation}
\mathbf{a}^{(L)} = g(\mathbf{z}^{(L)}) \equiv \hat{\mathbf{y}}.
\end{equation}
\newpage
\subsection*{Cost function}
For $m$ instances in the training set, the regularized cost function of the network is defined as 

\begin{equation}
J(\{\boldsymbol{\Theta_i}\}) = -\frac{1}{m}\sum_m\mathbf{y}_m\cdot\log(\hat{\mathbf{y}_m}) - (1 -\mathbf{y}_m)\cdot\log(1 -\mathbf{\hat{y}}_m) + \frac{\alpha}{2m}\sum_j\sum_{i=2}\sum_{l=1}\left(\Theta^{(j)}_{il}\right)^2
\end{equation}

\subsection*{Backpropagation}

Given a training instance $\mathbf{y}$, the error of the last layer ($j=L$) is set to

\begin{equation}
\boldsymbol{\delta}^{(L)} = \hat{\mathbf{y}} - \mathbf{y}.
\end{equation}

For $L > j >1$, the error associated to each layer is given by

\begin{equation}
\boldsymbol{\delta}^{(j)} = \left(\boldsymbol{\tilde{\Theta}}^{(j)}\right)^{\rm T} \boldsymbol{\delta}^{(j+1)} \circ g'(\mathbf{z}^{(j)}),
\end{equation}

where $\circ$ denotes element-wise multiplication and $\boldsymbol{\tilde{\Theta}}^{(j)}$ corresponds to the weight matrix $j$ without the first column. This is needed to exclude the bias unit from the backpropagation, which is not connected to the input unit.

The error associated with the weight matrix $j$ is given by a matrix $\mathbf{D}^{(j)}$, which is defined as

\begin{equation}
D^{(j)}_{il} = \frac{\partial J}{\partial \Theta^{(j)}_{il}}.
\end{equation}

Using the error of each individual layer, the error associated with the weight matrix $j$ can be defined as

\begin{equation}
D^{(j)}_{il} = \begin{cases} \frac{1}{m} \Delta^{(j)}_{il} \qquad \text{  if } i=0 \\ \frac{1}{m} \Delta^{(j)}_{il} + \frac{\alpha}{m} \Theta^{(j)}_{il} \qquad \text{else}.
\end{cases}
\end{equation}
where the matrix $\boldsymbol{\Delta}^{(j)}$ accumulates the errors when going through the training set with $m$ instances

\begin{equation}
\boldsymbol{\Delta}^{(j)} = \boldsymbol{\Delta}^{(j)} + \boldsymbol{\delta}^{(j+1)}\left(\mathbf{a}^{(j)}\right)^{\rm T}.
\end{equation}
\end{document}
